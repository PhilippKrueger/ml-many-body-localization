\documentclass[reprint,amsmath,amssymb,aps,prb]{revtex4-2}

\usepackage{graphicx}% Include figure files
\usepackage{dcolumn}% Align table columns on decimal point
\usepackage{bm}% bold math
\usepackage{hyperref}% add hypertext capabilities
\usepackage{xcolor}

\usepackage{listings} % insert code fragments

\definecolor{codegreen}{rgb}{0,0.6,0}
\definecolor{codegray}{rgb}{0.5,0.5,0.5}
\definecolor{codepurple}{rgb}{0.58,0,0.82}
\definecolor{backcolour}{rgb}{0.95,0.95,0.92}

\lstdefinestyle{mystyle}{
    backgroundcolor=\color{backcolour},   
    commentstyle=\color{codegreen},
    keywordstyle=\color{magenta},
    numberstyle=\tiny\color{codegray},
    stringstyle=\color{codepurple},
    basicstyle=\ttfamily\footnotesize,
    breakatwhitespace=false,         
    breaklines=true,                 
    captionpos=b,                    
    keepspaces=true,                 
    numbers=left,                    
    numbersep=5pt,                  
    showspaces=false,                
    showstringspaces=false,
    showtabs=false,                  
    tabsize=2
}

\lstset{style=mystyle}

\begin{document}

%\include{guidelines}

\title{Machine Learning of Many Body Localization}

\author{Philipp Krüger}

\date{\today}% It is always \today, today,
             %  but any date may be explicitly specified

\begin{abstract}
A paper usually includes an abstract, a concise summary of the work covered at length in the main body of the paper. Please also write a short abstract of your project.
%todo About 5% of article length & < 500 words
\end{abstract}


\maketitle

\section{Introduction}%todo: Action title

Introduce concepts: Exact Diagonalization, 

areal Density Matrix:
\url{http://www.thphys.nuim.ie/staff/jvala/Lecture_9.pdf}

\begin{figure}[h!]
	\includegraphics[width=0.99\linewidth]{figures/reduced_density_matrix.jpg}
	\caption{Example of a figure~\cite{Orus2013}.}
	\label{fig:adm}
\end{figure}

direct density matrix:
(DOI: 10.1103/PhysRevB.99.054208 says 
Instead of dividing the system into two subsystems A
and B to calculate the reduced density matrix of an eigenstate rho a and using the entanglement spectrum as the training data set 34,35, we directly feed the
probability density of the eigenstate psi i computed in the
spin basis to the machines as the training data set. The
reason for doing so is that, although by preprocessing
the training data can reduce the dimension and filter out
redundant information, useful information contained in
the wavefunction of the entire system can also be lost.)

Conclusion: => areal

, Neural Network, CNN

Outcome expectation: J<g: adiabatically connected to trivial state, J>g ordered phase

Is scaling important?

Review Literature on task

\section{Materials and Methods}%todo: Action title

Explain Flow with figure

\begin{figure}[h!]
	\includegraphics[width=0.99\linewidth]{figures/flowchart.jpg}
	\caption{Example of a figure~\cite{Orus2013}.}
	\label{fig:example}
\end{figure}

Fig.~\ref{fig:example}

Explain metrics and errors and why they are used. Which ml models are used and why?

Hyperparameters? Amount of layers?

\section{Results}%todo: Action title

\subsection{Generation of density matrix training set}

Plots: 
What is computationally realizable in 1h concerning time?
The training set was sufficiently large enough 

We only need M Eigenstates

This is how corresponding density matrices look like

This will be our parameter space for n, L

\subsection{Prediction of extended vs localized phase}

Training and testing scores

\subsection{$W_c$ analysis}

Those are our $W_c$ depending on n, L.

\section{Conclusion}%todo: Action title

$W_c$ depends on n, L (yes/no).

$W_c$ prediction coincides with the expectation (yes/no)

$W_c$ is dependent on these and that effects => scaling analysis? (yes/no)

Citations are numerical\cite{epr}, some more citations~\cite{feyn54,Bire82,Berman1983,witten2001,Davies1998}. 


\bibliography{bibsamp}% Produces the bibliography via BibTeX.


\appendix


\begin{widetext}
\section{Code listing} \label{app:codes}
Please copy your code in the appendix.
\begin{lstlisting}[language=Python]
"""

Description

"""

import numpy as np

code
\end{lstlisting}
\end{widetext}



\end{document}
