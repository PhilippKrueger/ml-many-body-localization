\documentclass[reprint,amsmath,amssymb,aps,prb]{revtex4-2}

\usepackage{graphicx}% Include figure files
\usepackage{dcolumn}% Align table columns on decimal point
\usepackage{bm}% bold math
\usepackage{hyperref}% add hypertext capabilities
\usepackage{xcolor}

\usepackage{listings} % insert code fragments

\definecolor{codegreen}{rgb}{0,0.6,0}
\definecolor{codegray}{rgb}{0.5,0.5,0.5}
\definecolor{codepurple}{rgb}{0.58,0,0.82}
\definecolor{backcolour}{rgb}{0.95,0.95,0.92}

\lstdefinestyle{mystyle}{
    backgroundcolor=\color{backcolour},   
    commentstyle=\color{codegreen},
    keywordstyle=\color{magenta},
    numberstyle=\tiny\color{codegray},
    stringstyle=\color{codepurple},
    basicstyle=\ttfamily\footnotesize,
    breakatwhitespace=false,         
    breaklines=true,                 
    captionpos=b,                    
    keepspaces=true,                 
    numbers=left,                    
    numbersep=5pt,                  
    showspaces=false,                
    showstringspaces=false,
    showtabs=false,                  
    tabsize=2
}

\lstset{style=mystyle}

\begin{document}


\title{Project title}

\author{Your Name}

\date{\today}% It is always \today, today,
             %  but any date may be explicitly specified

\begin{abstract}
A paper usually includes an abstract, a concise summary of the work covered at length in the main body of the paper. Please also write a short abstract of your project.
\end{abstract}


\maketitle


\section{Guidelines}

Please write a short paper about your project. There are no strict length limits, but please write \textbf{at least two pages in the layout of this template} (including figures, excluding references and code listings). Please structure your paper in a scientific way, and include your references and your code. There are \LaTeX packages you can use to preserve the indentation of your code, e.g. the \texttt{listings} package which is demonstrated in the Appendix~\ref{app:codes}.

This sample document makes use of of REV\TeX~4.2, therefore you will need to install it to be able to compile this document yourself. Further information can be found in the REV\TeX~4.2
documentation included in the distribution or available at
\url{http://journals.aps.org/revtex/}.


\subsection{Example citations}
By default, citations are numerical\cite{epr}, some more citations~\cite{feyn54,Bire82,Berman1983,witten2001,Davies1998}. 

\subsection{Exampe figure}
Including and referring to figures is as usual, see for instance Fig.~\ref{fig:example}
\begin{figure}
\includegraphics[width=0.99\linewidth]{cartoon.png}
\caption{Example of a figure~\cite{Orus2013}.}
\label{fig:example}
\end{figure}

\bibliography{bibsamp}% Produces the bibliography via BibTeX.


\appendix


\begin{widetext}
\section{Code listing} \label{app:codes}
Please copy your code in the appendix.
\begin{lstlisting}[language=Python]
"""

Module to generate the Hamiltonian of the transverse field Ising model.

H = -J sum_i sigma^x_i sigma^x_{i+1} - g sum_i sigma^z i.

Used in the solution of exercise 5.1

"""

import numpy as np
import scipy
from scipy import sparse
import scipy.sparse.linalg
import matplotlib.pyplot as plt

Id = sparse.csr_matrix(np.eye(2))
Sx = sparse.csr_matrix([[0., 1.], [1., 0.]])
Sz = sparse.csr_matrix([[1., 0.], [0., -1.]])
Splus = sparse.csr_matrix([[0., 1.], [0., 0.]])
Sminus = sparse.csr_matrix([[0., 0.], [1., 0.]])


def singesite_to_full(op, i, L):
    op_list = [Id]*L  # = [Id, Id, Id ...] with L entries
    op_list[i] = op
    full = op_list[0]
    for op_i in op_list[1:]:
        full = sparse.kron(full, op_i, format="csr")
    return full


def gen_sx_list(L):
    return [singesite_to_full(Sx, i, L) for i in range(L)]


def gen_sz_list(L):
    return [singesite_to_full(Sz, i, L) for i in range(L)]


def gen_hamiltonian_periodic(sx_list, sz_list, g, J=1.):
    """ assumes periodic boundery conditions """
    L = len(sx_list)
    H = sparse.csr_matrix((2**L, 2**L))
    for j in range(L):
        H = H - J *( sx_list[j] * sx_list[(j+1)%L])
        H = H - g * sz_list[j]
    return H
\end{lstlisting}
\end{widetext}



\end{document}
